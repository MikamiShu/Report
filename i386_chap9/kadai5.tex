\chapter{アセンブリ言語で直接アルゴリズムを記述する利点は存在するか}
\section{実験の目的}
アセンブリ言語で直接アルゴリズムを記述することによる利点が存在するか明らかにする.

\section{実験の方法}
アセンブリ,Java言語で記述した場合のメリット,デメリットをまとめ,比較する.

\section{実験結果}
以下の表が得られた.
\begin{table}[h]
  \begin{tabular}{c|clll}
   / &  アルゴリズムの記述 & コード量 & コードの複雑さ & 実行時間\\ \hline
   Java & 可能 & 29行 & 直感的 & $O(n^2)$\\
   アセンブリ & 可能 & 57行 & 難解 & $O(n^2)$\\
  \end{tabular}
\end{table}

\section{考察}
実験結果より,アセンブリ言語には高級言語と比較しても利点など無いように思えるが,\ref{fig:実行時間の比較}より,データ数が大きくなると,実行時間に明らかな差が出てくることがわかる.これらを総合して考えると,アセンブリ言語で直接アルゴリズムを記述することにも利点は存在するが,デメリットのほうが大きいと考察することができる.だが,処理速度という一点のみを重視するのであれが話は別である.\\
 \ \ また,選択ソートアルゴリズムに限った実験しか行っていないため,その他アルゴリズムにもこれが当てはまるかは定かでない.
